\section{Speed Enhancements}
We put a lot of effort on speeding up the our system's pipeline.
The most effective optimization was parallelizing the feature extraction by extracting each image's features in a separate process and then collecting all the features before training.

Processes are quite heavy, but python threads are totally useless, thanks to \texttt{GIL}'s locking mechanism.
We believe that if we port the code to another language, the execution time would be much lower using threads and manual memory allocation.

Python lists are known to be very slow, so we replaced them all with numpy arrays, and allocated most of the needed memory ahead before the system starts operating on the images.
A quite speed gain came from fine tuning \texttt{skimage} and \texttt{OpenCV} parameters.

We tried to use \texttt{Numba} and \texttt{Cython} to optimize the execution time but they didn't have an effect.
It was probably because most of the code calls numpy, \texttt{skimage} and \texttt{OpenCV}, which are all written in C and well optimized for memory and CPU.